% Options for packages loaded elsewhere
\PassOptionsToPackage{unicode}{hyperref}
\PassOptionsToPackage{hyphens}{url}
%
\documentclass[
]{article}
\usepackage{amsmath,amssymb}
\usepackage{lmodern}
\usepackage{ifxetex,ifluatex}
\ifnum 0\ifxetex 1\fi\ifluatex 1\fi=0 % if pdftex
  \usepackage[T1]{fontenc}
  \usepackage[utf8]{inputenc}
  \usepackage{textcomp} % provide euro and other symbols
\else % if luatex or xetex
  \usepackage{unicode-math}
  \defaultfontfeatures{Scale=MatchLowercase}
  \defaultfontfeatures[\rmfamily]{Ligatures=TeX,Scale=1}
\fi
% Use upquote if available, for straight quotes in verbatim environments
\IfFileExists{upquote.sty}{\usepackage{upquote}}{}
\IfFileExists{microtype.sty}{% use microtype if available
  \usepackage[]{microtype}
  \UseMicrotypeSet[protrusion]{basicmath} % disable protrusion for tt fonts
}{}
\makeatletter
\@ifundefined{KOMAClassName}{% if non-KOMA class
  \IfFileExists{parskip.sty}{%
    \usepackage{parskip}
  }{% else
    \setlength{\parindent}{0pt}
    \setlength{\parskip}{6pt plus 2pt minus 1pt}}
}{% if KOMA class
  \KOMAoptions{parskip=half}}
\makeatother
\usepackage{xcolor}
\IfFileExists{xurl.sty}{\usepackage{xurl}}{} % add URL line breaks if available
\IfFileExists{bookmark.sty}{\usepackage{bookmark}}{\usepackage{hyperref}}
\hypersetup{
  pdftitle={Projeto Crime Rio},
  pdfauthor={Gabriela Alves},
  hidelinks,
  pdfcreator={LaTeX via pandoc}}
\urlstyle{same} % disable monospaced font for URLs
\usepackage[margin=1in]{geometry}
\usepackage{graphicx}
\makeatletter
\def\maxwidth{\ifdim\Gin@nat@width>\linewidth\linewidth\else\Gin@nat@width\fi}
\def\maxheight{\ifdim\Gin@nat@height>\textheight\textheight\else\Gin@nat@height\fi}
\makeatother
% Scale images if necessary, so that they will not overflow the page
% margins by default, and it is still possible to overwrite the defaults
% using explicit options in \includegraphics[width, height, ...]{}
\setkeys{Gin}{width=\maxwidth,height=\maxheight,keepaspectratio}
% Set default figure placement to htbp
\makeatletter
\def\fps@figure{htbp}
\makeatother
\setlength{\emergencystretch}{3em} % prevent overfull lines
\providecommand{\tightlist}{%
  \setlength{\itemsep}{0pt}\setlength{\parskip}{0pt}}
\setcounter{secnumdepth}{-\maxdimen} % remove section numbering
\usepackage{booktabs}
\usepackage{longtable}
\usepackage{array}
\usepackage{multirow}
\usepackage{wrapfig}
\usepackage{float}
\usepackage{colortbl}
\usepackage{pdflscape}
\usepackage{tabu}
\usepackage{threeparttable}
\usepackage{threeparttablex}
\usepackage[normalem]{ulem}
\usepackage{makecell}
\usepackage{xcolor}
\ifluatex
  \usepackage{selnolig}  % disable illegal ligatures
\fi

\title{Projeto Crime Rio}
\author{Gabriela Alves}
\date{14/07/2021}

\begin{document}
\maketitle

\hypertarget{criminalidade-no-rio-de-janeiro}{%
\section{Criminalidade no Rio de
Janeiro}\label{criminalidade-no-rio-de-janeiro}}

Este estudo tem como objetivo criar um ranking com base em dados sobre a
criminalidade do Estado do Rio de Janeiro. A proposta de criação deste
ranking se dá por meio da Análise Fatorial por Componentes Principais,
uma técnica não supervisionada de machine learning que utiliza variáveis
quantitativas para criar fatores que são responsáveis por capturar a
variância compartilhada entre as variáveis e estes fatores são
ortogonais entre si. Esta base de dados contém 53 variáveis
quantitativas e pode ser encontrada no
\href{https://www.ispdados.rj.gov.br:4432/estatistica.html}{link}.

\begin{verbatim}
##            tidyverse                knitr           kableExtra 
##                 TRUE                 TRUE                 TRUE 
##                  car                  rgl            gridExtra 
##                 TRUE                 TRUE                 TRUE 
## PerformanceAnalytics             reshape2            rayshader 
##                 TRUE                 TRUE                 TRUE 
##                psych               pracma              polynom 
##                 TRUE                 TRUE                 TRUE 
##                rqPen              ggrepel               plotly 
##                 TRUE                 TRUE                 TRUE 
##           factoextra 
##                 TRUE
\end{verbatim}

\hypertarget{anuxe1lise-fatorial-por-componentes-principais}{%
\subsection{Análise fatorial por componentes
principais}\label{anuxe1lise-fatorial-por-componentes-principais}}

A análise fatorial é uma técnica de aprendizado de máquina não
supervisionada. É uma técnica multivariada que utiliza apenas variáveis
quantitativas/métricas e essas variáveis tem um coeficiente de
correlação alto entre si. O objetivo da análise fatorial é a extração de
fatores a partir de uma matriz de correlação gerada dessas variáveis (e
por isso tais variáveis devem ser métricas). Os fatores representam o
comportamento conjunto das variáveis originais, e pode ser visto como um
agrupamento de variáveis.

O método de extração por componentes principais é o mais utilizado para
análise fatorial, este método extrai os fatores a partir das combinações
lineares das variáveis originais.

Esses fatores são ortogonais entre si, isto é, possui correlação 0 entre
si, e são capazes de representar o comportamento conjunto das variáveis
originais, e pode ajudar na observação de comportamentos que antes não
eram possíveis de serem vistos, e portanto pode-se entender como um
agrupamento de variáveis.

A análise fatorial por componentes principais tem três grandes usos: -
1) Redução da dimensionalidade da base dados por meio do agrupamento de
variáveis. - 2) Confirmação de constructos pré determinados. - 3)
Elaboração de rankings por meio da criação de indicadores. Sendo a
elaboração de rankings uma de suas possibilidades é portanto por este
motivo que será utilizada a análise fatorial por componentes principais
nesta base de dados, que tem por objetivo a criação de um ranking dos
municípios do Rio de Janeiro por meio de variáveis relacionadas à
segurança pública.

\hypertarget{ponderauxe7uxe3o-arbitruxe1ria}{%
\subsection{Ponderação
arbitrária}\label{ponderauxe7uxe3o-arbitruxe1ria}}

No entanto, faço voz para que não se confunda aqui uma variável
quantitativa com uma qualitativa que seja expressada de forma numérica,
como uma variável binária que representa evento e não evento de algo, ou
uma Escala deLikert. Tanto a variável binária de evento quanto a Escala
de Likert são variáveis que expressão categorias, logo não são
numéricas, não é possível obter delas por exemplo, medidas de tendência
central,dispersão ou correlação. E uma grande ressalva para a Escala de
Likert, ainda que seja expressa em números, tal variável tem apenas
valor semântico, e valor semântico não se mede, qual é a diferença
numérica deruim para péssimo? Realizar uma análise fatorial com tais
variáveis resultaria em um problema de \textbf{Ponderação Arbitária},
onde se estará assumindo pesos numéricos para variáveis quali, onde por
exemplo, se estívéssemos falando de raça/cor, e dizemos que a cor branca
é representada por 4 e a preta é 2, estamos dizendo com isso que a cor
branca vale 2x mais que a preta, e podemos levar preconceito para nossa
análise, por isso o cuidado ao trabalhar com variáveis qualitativas.

\hypertarget{apresentando-a-base-de-dados}{%
\subsection{Apresentando a base de
dados}\label{apresentando-a-base-de-dados}}

Ao todo esta base contem 92 observações e 54 variáveis, onde apenas a
primeira, fmun\_cod é qualitativa e representa o nome dos municípios do
Estado do Rio de Janeiro. As restantes variáveis são todas quantitativas
e representam diferentes tipos de crimes. Para este estudo foi
selecionado os dados de 2020, que originalmente são apresentados no
formato mensal, no entanto para observar o ano e evitar a escolha de um
mês que possa ser atípico, esses dados mensais foram somados e por isso
temos apenas 92 observações.

\url{https://www.geeksforgeeks.org/sum-of-rows-based-on-column-value-in-r-dataframe/}
\url{https://www.biostars.org/p/303219/}

\end{document}
